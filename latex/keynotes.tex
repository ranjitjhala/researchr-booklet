\header{Keynotes}{}{}{ICFP Keynotes}
\label{Keynotes}

\def\talktitle#1{\subsection*{#1}}
\def\speaker#1#2{\begin{flushleft} #1 (#2) \end{flushleft}}

\talktitle{Compositional Creativity: Some Principles for Talking to Computers}
\speaker{Chris Martens}{North Carolina State University}
\def\talkabstract{\noindent \textbf{Abstract:}~}
\def\bio{\medskip\noindent \textbf{Bio:}~}

\talkabstract
\emph{Generativity} is an increasingly popular and useful concept, referring to a machine’s ability to respond to user input with new constructions not foreseen by the programmer. Yet increasingly, people treat computational systems as unknowable black-box systems, writing off the possibility of forming mental models that allow a collaborative relationship between human insight and fast computation. 

I argue for the efficacy of transparent, compositional semantics for collaborating with virtual agents and deriving insights from system models. Having built systems based on automated reasoning for linear logic and epistemic modal logic, we can formalize notions of belief, intention, and action, in order to create virtual agents that behave in ways that humans can reason about based on intuitions about goal-driven behavior. For example, some of Grice’s maxims of conversation can be seen as derivable consequences of these principles. Ongoing work includes applying these formalisms to the tasks of navigating unknown rule systems in virtual environments, social skills training, and generative storytelling.

\bio
Chris Martens is an Assistant Professor at North Carolina State University, where she directs the Principles of Expressive Machines (POEM) lab. She was a postdoctoral researcher with UC Santa Cruz’s Expressive Intelligence Studio after completing her Ph.D. with the Principles of Programming group at Carnegie Mellon in 2015. At CMU she worked on logical frameworks, dependent type theory, and representing narrative structure with linear logic. Her current research activities include applying automated reasoning and compositional semantics to the authorship of generative systems, games, and narrative, as well as designing accessible tools for system modeling and logical specification.

\medskip

\talktitle{Ten Years of Clojure---FP Out of the Box}
\speaker{Rich Hickey}{Cognitect Inc.}

\talkabstract
Since its first release in 2007, Clojure has seen considerable adoption in industries as
diverse as Fortune 5 retail, international banking and finance, climate science and the
latest startups. Clojure has tens of thousands of users and is a top 20 language in the
RedMonk 2017 index. This is somewhat unexpected (especially by its author!)\ given that
Clojure is a Lisp dialect emphasizing functional programming. What happened?

This talk examines the objectives of Clojure and its initial and ongoing design decisions.
Moreover, we can now examine how that's worked out---which features are valued by its users
and how it is employed to build systems. We will briefly touch on Datomic, a functional
database system written in Clojure by the same authors that completes the model of
programming Clojure espouses. Finally we'll look at what challenges remain for programmers
and how Clojure might grow to address them.

\bio
Rich Hickey is the author of Clojure, and the designer of Datomic. Rich has 30 years of
professional experience in all facets of software development. Rich has worked on
scheduling systems, broadcast automation, audio analysis and fingerprinting, database
systems, yield management, exit poll systems, and machine listening, in a variety of
languages.

\newpage
