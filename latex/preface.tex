\header{Preface}{}{}{Message from the Chairs}
\label{Preface}

\noindent
Welcome to Oxford, and to ICFP 2017, the \textit{22\textsuperscript{nd} ACM SIGPLAN International Conference on Functional Programming}!

ICFP provides a forum for sharing and discussing the latest work on the art and science of functional programming. We hope that you will enjoy the conference, including not only the technical presentations on a wide spectrum of topics, but also the opportunities for learning from and interacting with many researchers, developers, and students from around the world.

This year's call for papers resulted in a near-record of 127 submissions, including 20 functional pearls and 4 experience reports. From these, the program committee selected 44 papers for presentation at the conference, including 6 pearls and 2 experience reports. In addition, the technical program includes invited keynotes by Chris Martens and Rich Hickey.

Following recent practice, ICFP 2017 employed a lightweight double-blind reviewing process, with each paper receiving three or more reviews from members of the program committee and a pool of external reviewers. Initial reviews were made available to authors, many of whom provided useful feedback and clarification during a 72-hour author response period. This was followed by online discussion and an in-person meeting of the program committee in Portland, Oregon, where the final selection of papers was made. Papers whose author list included a member of the program committee were not discussed until all other decisions had been made, and were held to a higher standard; ultimately, 4 of the 12 papers in this category were accepted.

All of the papers accepted for ICFP 2017 are being published in the inaugural edition of a new journal, Proceedings of the ACM on Programming Languages (PACMPL), which is a Gold Open Access journal publishing research on all aspects of programming languages, from design to implementation and from mathematical formalisms to empirical studies. 

One of the more visible changes resulting from our participation in PACMPL is the new, single-column document format. The transition from the previous two-column format was not always easy, either for authors or reviewers. We thank everyone for their patience and goodwill throughout multiple iterations of this process, and we hope that the community will benefit from resulting improvements in readability and accessibility.

A less obvious impact of the move to PACMPL was the introduction of a new, two-phase selection and reviewing process, further enhancing the already rigorous approach that had been used in prior years. Concretely, this allowed for papers to be ``conditionally accepted'' at the end of the program committee meeting, and for the associated reviews to be annotated with a list of ``mandatory revisions.''  Authors of conditionally accepted papers then had approximately five weeks to revise and submit their papers for a second and final reviewing phase, including a cover letter to explain how the mandatory revisions had been addressed. In total, 30 papers were conditionally accepted, with mandatory revisions ranging from minor to more substantial but all judged to be feasible given the limited five week period for revision. While it is difficult to quantify the overall impact of this process, there were many cases where the extended opportunities for interaction between authors and reviewers helped to improve quality in both technical details and presentation. Ultimately, all conditionally accepted papers were accepted for inclusion in the conference.

Another new feature for ICFP this year that will be apparent from the papers in the proceedings is the introduction of an artifact evaluation process, with selected papers receiving a corresponding ``badge'' or ``seal of approval'' on their first page. Artifact evaluation, which supports future researchers in reproducing and building on current work, has proved to be a valuable component of other conferences, and we are confident that the ICFP community will benefit in a similar way. Although the process is optional, all authors of accepted papers were invited to prepare and submit artifacts to accompany and support their paper. These items were reviewed by an artifact evaluation committee that also provided feedback to authors to help improve the quality of submitted artifacts. The committee received 31 artifact submissions, all of which were accepted, and 27 of the associated papers are further badged as having a publicly available artifact.

As usual, the main conference is complemented by a range of affiliated events, including twelve co-hosted conferences, workshops or symposia, as well as the ICFP Programming Contest and the Student Research Competition, with results for both announced during the conference.  In addition to technically focussed workshops on a broad range of topics, we are proud to include the SIGPLAN \textit{Programming Languages Mentoring Workshop} (PLMW) at ICFP. The purpose of this mentoring workshop is to encourage senior undergraduate and beginning graduate students to pursue careers in programming language research, and to engage them in a process of imagining how they might contribute to the world. A novelty this year is that ICFP will run in parallel with FSCD 2017, the \textit{Second International Conference on Formal Structures for Computation and Deduction}, with joint lunch and coffee breaks and the same overall schedule structure so that registrants at either conference can attend talks at the other. 

It is hard to overstate the fundamental importance of community to the good health and success of ICFP. Of course, this includes the authors and developers that share their work, and the attendees who provide a stimulating environment for discussion and debate. But special recognition is due for the many volunteers who---even when they are already very busy with other commitments---still step up to take on new responsibilities and roles in support of the conference. We are deeply humbled, impressed, and grateful for their commitment, hard work, and expertise. In particular, we would like to acknowledge the program committee, the external reviewers, and the artifact evaluation committee for their thorough and thoughtful reviews; Annabel Satin and Marta Zampollo, for their excellent arrangements; the members of the ICFP steering committee, for their long term stewardship and dedication to the success of the conference; Ryan Trinkle, for liaising with our industrial partners and sponsors; David Christiansen and Andres Löh for their leadership in organizing the associated workshops; Neelakantan Krishnaswami, Dan Licata, and Brigitte Pientka for chairing PLMW; Ilya Sergey for running the Student Research Competition; Sam Lindley for coordinating the ICFP Programming Contest; Lindsey Kuper for making sure that the ICFP community is informed and engaged; José Calderón for managing the process of recording and posting videos for many of the talks at ICFP and associated events; Dirk Beyer and Conference Publishing Consulting for compiling the proceedings; the PACMPL Editorial Board and the staff at ACM headquarters---especially Philip Wadler, Michael Hicks, Matthew Fluet, Scott Delman, Craig Rodkin, and Laura Lander---for guiding and supporting the transition to PACMPL; Ted Cooper and Larry Diehl for their assistance during the program committee meeting; and last, but not least, Yosuke Fukuda, Yuki Nishida, and Jakub Zalewski for organizing and leading the team of student volunteers that will be working to keep everything running smoothly during the conference.

We are indebted to our partners who made it possible to keep the cost of registration reasonable, and who provided support for students who would not have been able to attend without financial aid. We are also very grateful for the generous support of ACM and ACM SIGPLAN, including their commitment to PACMPL's Gold Open Access policy, making high-quality, peer-reviewed scientific research available without restrictions on access or (re-)use. The generosity of our supporters is key in helping our community to grow and thrive.

Welcome again to ICFP. We hope you enjoy the conference, the affiliated events, the location, and the opportunities that it all provides to meet new people and, for those who have attended previously, to reconnect with colleagues and friends!

\begin{flushright}
\textit{Jeremy Gibbons, University of Oxford, UK} \\
ICFP 2017 General Chair 
\medskip \\
\textit{Mark P. Jones, Portland State University, USA} \\
ICFP 2017 Program Chair
\medskip \\
\textit{Ryan R. Newton, Indiana University, USA} \\
\textit{Matthew Flatt, University of Utah, USA} \\
ICFP 2017 Artifact Evaluation Committee Co-Chairs
\end{flushright}


\newpage
